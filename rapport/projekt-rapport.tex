
\documentclass[a4paper]{article}

\usepackage{graphicx}
\usepackage{listings}
\usepackage{indentfirst}
\usepackage{float}
%\usepackage[T1]{fontenc}                % F�r svenska bokst�ver
%\usepackage[swedish]{babel}             % F�r svensk avstavning och svenska
                                        % rubriker (t ex "inneh�llsf�rteckning)
\title{Programming Project, Network programming}
\author{Tim Dolck dat11tdo@student.lu.se \\ 
  Julian Kron\'{e} dat11jkr@student.lu.se \\
Christopher Nilsson dat11cni@student.lu.se \\
Anton Lin, dic13ali@student.lu.se \\
Computer science, LTH}
%\date{}           % Blir dagens datum om det utel�mnas

\begin{document}
\maketitle
\newpage
\section{Background}
The objective of this project is to create an application where the project members' knowledge in network programming is utilized to as great an extent as possible. 
The project members' love for arcade- and retro-esque games in combination with the desire to create a kickass network application has resulted in a multiplayer game with gameplay influences from Space Invaders and the look and feel of Asteroids.     
 
The network communication is the central part of the application. The plan is to have up to four players to be able to view each other in real time and also be able to interact with eachother. 
 
All of the project members have previous experience in Java, which is why it became the programming language of choice.
Through extensive research, the choice of game library fell upon Slick2D, a lightweight and easy to use extension of the popular game library \"LWJGL(Lightweight Java Game Library\".  

\section{Requirements}

Develop a game in Java that can handle multiple players. Connected through different devices over the local network.
The game should consist of three separate parts; client, server, and game. Starting the application, 
and setting up a game, should be as convenient as possible for the user. When creating a new game, 
a server should be set up automatically, parallel a client that connects to it.
When the game is set up, the players are entered into a lobby waiting for all players to be ready.

The game should be a simplified version of asteroids. The player can only move horizontally, and shoot upwards.
The creeps fall down, attacking the players. The cross-player interaction consist of the possibility to send 
additional creeps, making it harder for them to survive.

\subsection{Functionality}
\begin{itemize}
  \item Set up a server that the clients can connect to.
  \item Set up clients that can connect to the server and receive information from the other clients.
  \item Server support up to 4 connected clients.
  \item Minimize traffic by using bytes as messages.
  \item Parsers between the bytes and the actual information.
  \item Develop a stable and obvious protocol.
  \item Receive and parse user inputs.
  \item Application lifecycle (Splashscreen, Main menu, lobby, game, game over).
  \item Individual player lifecycles (Alive, dead).
  \item See player movements, creeps, bullets, credits, levels.
  \item Send creeps to other players.
  \item Remove irrelevant objects.
  \item End game when only one player is alive, and display the winner.
\end{itemize}

\subsection{Quality}
\begin{itemize}
  \item Stable network connections.
  \item Low latency, the game should appear to run in real-time.
  \item The game should run in 60fps.
\end{itemize}
\section{Model}

\section{Manual}

\subsection{Game}

SpaceDefence is a game where up to four players play together. 
Each player has its own board where the creeps arrive and is killed.
All other players connected to the game is shown in their own boards with their own creeps.
To interact with the other players it is possible to send creeps the other players for a small cost. 
Each player can also upgrade their fire rate and income rate.

\subsection{Controls}

Each player controls it own ship. 
To move the ship the direction keys left and right is used. The movements is limited to each players own board.

To shoot the space key is pressed. This shoots a bullet that may kill creeps.
Bullets can only be fired with a maximum fire rate. This depends on the players current level.

To upgrade the ship (fire rate and income rate) the z-key is pressed. This is only possible when enough credits has been collected.

To send creeps to the opponents the 1, 2, 3 keys are used.
1 sends one creep, 2 sends five creeps and 3 sends ten creeps to all connected opponents.

\subsection{Running the program}

The program consist of one server and one client part. 
The usual use case is that the first client also starts up the server through the user-interface but the server can also be started separatly.
When the is downloaded from the website it consists of a compressed folder. Decompress it and go to the decompressed folder.
The decrompressed folder contains four .jar files. Three client jars which is platform dependent (Windows, OS X, Linux) and one for the server.
To start the server seperatly, start a terminal and write \texttt{java -jar server.jar}. 
When the server is started your ip is printed. This is used by the clients to connect to the server.
To start a client simply double click the provided client jar-file for your current platform. From within the gui you can also start a server.

\section{Evaluation}

\section{Programs}

\end{document}                 % The input file ends with this command.
