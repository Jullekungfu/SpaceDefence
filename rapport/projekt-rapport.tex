
\documentclass[a4paper]{article}

\usepackage{graphicx}
\usepackage{listings}
\usepackage{indentfirst}
\usepackage{float}
%\usepackage[T1]{fontenc}                % F�r svenska bokst�ver
%\usepackage[swedish]{babel}             % F�r svensk avstavning och svenska
                                        % rubriker (t ex "inneh�llsf�rteckning)
\title{Programming Project, Network programming}
\author{Tim Dolck dat11tdo@student.lu.se \\ 
  Julian Kron\'{e} dat11jkr@student.lu.se \\
Christopher Nilsson dat11cni@student.lu.se \\
Anton Lin, dic13ali@student.lu.se \\
Computer science, LTH}
%\date{}           % Blir dagens datum om det utel�mnas

\begin{document}
\maketitle
\newpage
\section{Background}

\section{Requirements}

Develop a game in Java that can handle multiple players. Connected through different devices over the local network.
The game should consist of three separate parts; client, server, and game. Starting the application, 
and setting up a game, should be as convenient as possible for the user. When creating a new game, 
a server should be set up automatically, parallel a client that connects to it.

The game should be a simplified version of asteroids. The player can only move horizontally, and shoot upwards.
The creeps fall down, attacking the players.

\subsection{Functionality}
\begin{itemize}
  \item Set up a server that the clients can connect to.
  \item Set up clients that can connect to the server and receive information from the other clients.
  \item Server support up to 4 connected clients.
  \item Minimize traffic by using bytes as messages.
  \item Parsers between the bytes and the actual information.
  \item Develop a stable and obvious protocol.
  \item Receive and parse user inputs.
  \item See player movements, creeps, bullets, credits, levels.
  \item Send creeps to other players.
  \item Remove irrelevant objects.
  \item Application lifecycle (Splashscreen, Main menu, lobby, game, game over).
  \item End game when only one player is alive.
  \item Show winner.
  \item Individual player lifecycles (Alive, dead).
\end{itemize}

\subsection{Quality}
\begin{itemize}
  \item Stable network connections.
  \item Low latency, the game should appear to run in real-time.
  \item The game should run in 60fps.
\end{itemize}

\section{Model}

\section{Manual}

\section{Evaluation}

\section{Programs}

\end{document}                 % The input file ends with this command.
