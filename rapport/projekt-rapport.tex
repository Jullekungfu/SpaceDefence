
\documentclass[a4paper]{article}

\usepackage{graphicx}
\usepackage{listings}
\usepackage{indentfirst}
\usepackage{float}
%\usepackage[T1]{fontenc}                % F�r svenska bokst�ver
%\usepackage[swedish]{babel}             % F�r svensk avstavning och svenska
                                        % rubriker (t ex "inneh�llsf�rteckning)
\title{Programming Project, Network programming}
\author{Tim Dolck dat11tdo@student.lu.se \\ 
  Julian Kron\'{e} dat11jkr@student.lu.se \\
Christopher Nilsson dat11cni@student.lu.se \\
Anton Lin, dic13ali@student.lu.se \\
Computer science, LTH}
%\date{}           % Blir dagens datum om det utel�mnas

\begin{document}
\maketitle
\newpage
\section{Background}

\section{Requirements}
Stable network connections
Possibilities to connect up to 4 players


\section{Model}

\section{Manual}

\subsection{Game}

SpaceDefence is a game where up to four players play together. 
Each player has its own board where the creeps arrive and is killed.
All other players connected to the game is shown in their own boards with their own creeps.
To interact with the other players it is possible to send creeps the other players for a small cost. 
Each player can also upgrade their fire rate and income rate.

\subsection{Controls}

Each player controls it own ship. 
To move the ship the direction keys left and right is used. The movements is limited to each players own board.

To shoot the space key is pressed. This shoots a bullet that may kill creeps.
Bullets can only be fired with a maximum fire rate. This depends on the players current level.

To upgrade the ship (fire rate and income rate) the z-key is pressed. This is only possible when enough credits has been collected.

To send creeps to the opponents the 1, 2, 3 keys are used.
1 sends one creep, 2 sends five creeps and 3 sends ten creeps to all connected opponents.

\subsection{Running the program}

\section{Evaluation}

\section{Programs}

\end{document}                 % The input file ends with this command.
